\chapter{HASIL YANG DIHARAPKAN}

\section{Hasil yang Diharapkan dari Penelitian}

\par Hasil yang diharapkan dari penelitian ini adalah sistem pendeteksi penggunaan Alat Pelindung Diri (APD) pada pekerja konstruksi ini dapat diimplementasikan secara \emph{real-time} di lokasi kontstruksi dan membantu proses pengawasan yang dilakukan oleh staf pengawas K3 di lapangan. Selain itu, dengan adanya sistem ini penulis juga berharap akan meningkatkan kedisiplinan pemakaian APD oleh pekerja konstruksi sehingga mengurangi kemungkinan cedera, sakit, maupun kematian di lokasi konstruksi jika terjadi kecelakaan kerja.

\section{Hasil Pendahuluan}

Sebagai pendahuluan, penulis telah melakukan beberapa bagian dari metodologi yang akan dilakukan dalam penilitian ini. Penulis diantaranya mencoba untuk melakukan \emph{preprocessing} dataset dan melakukan \emph{training} model YOLOv7 dengan custom dataset sebagai gambaran kedepannya bagaimana penilitian ini akan berjalan. \emph{Preprocessing} dataset dilakukan pada platform Roboflow, sedangkan untuk proses \emph{training} model YOLOv7 dilakukan pada platform Google Colab.