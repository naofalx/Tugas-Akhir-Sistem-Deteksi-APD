\addcontentsline{toc}{chapter}{ABSTRACT}

\begin{center}
  \large
  \textbf{ABSTRACT}
\end{center}

\begin{center}
  \large
  \textbf{PERSONAL PROTECTIVE EQUIPMENT (PPE) DETECTION SYSTEM ON CONSTRUCTION WORKERS}
\end{center}

% Menyembunyikan nomor halaman
% \thispagestyle{empty}

\begin{flushleft}
  \setlength{\tabcolsep}{0pt}
  \bfseries
  \begin{tabular}{lc@{\hspace{6pt}}l}
    Student Name / NRP & : & Muhammad Naofal Nirvana / 0721 19 4000 0066 \\
    Department         & : & Computer Engineering ELECTICS - ITS         \\
    Advisor            & : & 1. Reza Fuad Rachmadi, S.T., M.T., Ph. D    \\
                       &   & 2. Dr. I Ketut Eddy Purnama, ST., MT.       \\
  \end{tabular}
  \vspace{4ex}
\end{flushleft}
\textbf{Abstract}

% Isi Abstrak
Personal Protective Equipment (PPE) is a tool that has the ability to protect someone whose function is to isolate part or all of the body from potential hazards in the workplace. There are regulations that state the importance of and require the use of Personal Protective Equipment such as Regulation of the Minister of Manpower of the Republic of Indonesia NUMBER PER.08/MEN/VII/2010 concerning REGULATIONS FOR PERSONAL PROTECTIVE EQUIPMENT. Even though it is mandatory and regulated in a ministerial regulation, it does not guarantee that all field workers will wear the PPE. Companies working on construction projects usually have dispatched supervisors to ensure that all workers are wearing appropriate PPE. The placement of the supervisory staff itself has also been regulated in one of the regulations of the Ministry of Manpower of the Republic of Indonesia. However, the monitoring method used is still carried out manually by supervisory staff who has its limitations. The large area of the construction site and the large number of workers is a challenge for the supervisory staff as a human being to carry out their duties of supervising every worker in the area. Therefore, this study aims to design a system that can automatically detect workers wearing appropriate PPE and those wearing inappropriate PPE and triggering an alarm when the system detects workers wearing inappropriate PPE. System development will utilize the Convolutional Neural Network and the YOLOv7 object detection algorithm.

\vspace{2ex}
\noindent
\textbf{Keywords: \emph{Personal Protective Equipment, System, Detect, Convolutional Neural Network}}