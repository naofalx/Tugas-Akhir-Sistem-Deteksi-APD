\begin{center}
  \large
  \textbf{SISTEM PENDETEKSI PENGGUNAAN ALAT PELINDUNG DIRI (APD) PADA PEKERJA KONSTRUKSI}
\end{center}
\addcontentsline{toc}{chapter}{ABSTRAK}
% Menyembunyikan nomor halaman
\thispagestyle{empty}

\begin{flushleft}
  \setlength{\tabcolsep}{0pt}
  \bfseries
  \begin{tabular}{ll@{\hspace{6pt}}l}
    Nama Mahasiswa / NRP & : & Muhammad Naofal Nirvana / 0721 19 4000 0066 \\
    Departemen           & : & Teknik Komputer FTEIC - ITS                 \\
    Dosen Pembimbing     & : & 1. Reza Fuad Rachmadi, S.T., M.T., Ph. D    \\
                         &   & 2. Dr. I Ketut Eddy Purnama, ST., MT.       \\
  \end{tabular}
  \vspace{4ex}
\end{flushleft}
\textbf{Abstrak}

% Isi Abstrak
Alat Pelindung Diri (APD) adalah suatu alat yang mempunyai kemampuan untuk melindungi seseorang yang fungsinya mengisolasi sebagian atau seluruh tubuh dari potensi bahaya di tempat kerja. Terdapat peraturan yang menyatakan pentingnya dan mewajibkan penggunaan Alat Pelindung Diri seperti Peraturan Menteri Tenaga Kerja RI NOMOR PER.08/MEN/VII/2010 tentang PERATURAN ALAT PELINDUNG DIRI. Meskipun sudah diwajibkan dan diatur dalam peraturan menteri, tidak menjamin semua pekerja lapangan akan memakai APD tersebut. Perusahaan yang mengerjakan proyek konstruksi biasanya sudah mengerahkan pengawas untuk memastikan semua pekerja memakai APD yang sesuai ketentuan. Penempatan staf pengawas sendiri juga sudah diatur dalam salah satu peraturan Kementerian Ketenagakerjaan Republik Indonesia. Namun metode pengawasan yang digunakan masih dilakukan secara manual oleh staf-staf pengawas yang masih memiliki keterbatasan. Luasnya area lokasi kontstruksi dan jumlah pekerja yang sangat banyak menjadi tantangan tersendiri bagi staf pengawas sebagai seorang manusia untuk menjalankan tugasnya mengawasi setiap pekerja yang ada di area tersebut. Oleh karena itu, penelitian ini bertujuan untuk merancang suatu sistem yang bisa mendeteksi secara otomatis pekerja yang memakai APD lengkap dan yang memakai APD tidak lengkap serta memicu semacam alarm ketika sistem mendeteksi pekerja yang memakai APD yang tidak lengkap. Pengembangan sistem akan memanfaatkan \emph{Convolutional Neural Network} dan algoritma deteksi objek YOLOv7.

\vspace{2ex}
\noindent
\textbf{Kata Kunci: \emph{Alat Pelindung Diri, Sistem, Mendeteksi, Convolutional Neural Network}}